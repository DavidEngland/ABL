\section{Grid Sensitivity and Bulk Richardson Number Bias}
\label{sec:grid_bias}

\subsection{Bulk Richardson Number}

Atmospheric models discretize the vertical using finite layers. The \textbf{bulk Richardson number} across the first layer ($z_0$ to $z_1$) is
\begin{equation}
Ri_b = \frac{g}{\theta}\,\frac{(\theta_1-\theta_0)(z_1-z_0)}{(U_1-U_0)^2}.
\label{eq:rib_definition}
\end{equation}

For power-law or logarithmic profiles, the dynamically representative height is the \textbf{geometric mean}
\begin{equation}
z_g = \sqrt{z_0 z_1}.
\label{eq:zg}
\end{equation}

\subsection{Concavity and Underestimation}

By Jensen's inequality, if $Ri_g(z)$ is concave down over $[z_0,z_1]$, the layer average satisfies
\begin{equation}
Ri_b = \frac{1}{\Delta z}\int_{z_0}^{z_1} Ri_g(z)\,dz < Ri_g(z_g).
\label{eq:jensen}
\end{equation}

Thus, \textbf{$Ri_b$ systematically underestimates local stability} when $\Delta<0$. This causes turbulent diffusivities $K_m$ and $K_h$ to be overestimated, producing excessive mixing and degraded simulation of stable nocturnal boundary layers.

\subsection{Amplification Ratio}

Define the bias amplification ratio
\begin{equation}
B = \frac{Ri_g(z_g)}{Ri_b}.
\label{eq:bias_ratio}
\end{equation}

For $\Delta<0$ and coarse $\Delta z$, $B>1$ and can exceed 1.5--2.0 in strongly stable cases.
