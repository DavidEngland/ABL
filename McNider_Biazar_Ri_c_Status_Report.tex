% filepath: /Users/davidengland/Documents/GitHub/ABL/McNider_Biazar_Ri_c_Status_Report.tex
\documentclass[11pt]{article}
\usepackage{amsmath}
\usepackage{geometry}
\geometry{margin=1in}
\begin{document}

\begin{center}
\LARGE\textbf{Status Report: Dynamic Critical Richardson Number (Ri\_c*)}\\[4pt]
\normalsize For: R. McNider \& A. Biazar \\
Date: \today
\end{center}

\section*{Purpose}
Summarize current understanding (Ri.md material received), and request focused contributions from McNider (physics, mixing-length path) and Biazar (operational K-path, model integration) to develop, calibrate and validate a practical dynamic Ri\_c* for hybrid MOST/Ri closures.

\section*{Current status (brief)}
\begin{itemize}
\item Ri curvature and neutral invariant $\Delta$ documented; Jensen bias (bulk vs point) identified and quantified.
\item Proposed hybrid strategy: use Ri\_c* as regime classifier with MOST below and Ri-based closures above; blend/hysteresis specified.
\item Prototype formula and pseudocode exist in the repository (dynamic\_Ric\_strategy.md / dynamic critical notes).
\end{itemize}

\section*{Requested contributions (clear asks)}
\paragraph{McNider (physics lead)}
\begin{itemize}
\item Derive/justify physically motivated form(s) for Ri\_c*(t) using inversion strength, shear and turbulence memory; propose candidate functional forms and plausible coefficient ranges.
\item Develop mixing-length modification functions $l^*(Ri,Ri_c^*)$ (examples: $l/(1+a_l (Ri/Ri_c^*)^n)$). Provide analytic limits to preserve neutral curvature $2\Delta$.
\item Lead slope/terrain and intermittent turbulence tests (CASES-99, GABLS3-like LES); provide diagnostics and thresholds for hysteresis (Ri\_suppress/Ri\_restart).
\end{itemize}

\paragraph{Biazar (operational lead)}
\begin{itemize}
\item Prototype a multiplicative diffusivity path $K^*=K\cdot g_K(Ri,Ri_c^*)$ (exponential or power-law), implement in a small test branch of CMAQ/WRF or in the provided Python prototype.
\item Assess computational cost, backward compatibility, and namelist options; prepare guidance for WRF/CMAQ insertion points and necessary regression tests.
\item Lead air-quality impact experiments (urban case studies) to quantify sensitivity of nocturnal pollutant trapping to Ri\_c* tuning.
\end{itemize}

\section*{Data and benchmarks}
\begin{itemize}
\item Tower: SHEBA, ARM SGP, Cabauw (recommended for tuning $\alpha_\Gamma,\alpha_S,\alpha_T$).
\item LES: GABLS suite (GABLS1–3) for "truth" curvature and regime timing.
\item Urban: Dallas/DFW remote-sensing + tower for air-quality metrics.
\end{itemize}

\section*{Algorithm sketch}
\begin{enumerate}
\item Compute local diagnostics: $Ri_g(z_g)$, $Ri_b$, curvature $\partial^2Ri_g/\partial\zeta^2$, TKE (or proxy).
\item Compute Ri\_c*:
\[
Ri_c^* = Ri_{c,0}\Big[1+\alpha_\Gamma\frac{\Gamma}{\Gamma_{\text{ref}}}+\alpha_S\frac{S}{S_{\text{ref}}}+\alpha_T\frac{\text{TKE}_{\text{prev}}}{\text{TKE}_{\text{ref}}}\Big]
\]
with optional curvature term $\alpha_{\rm curv}|\partial^2Ri_g/(2\Delta)|$.
\item Regime logic: MOST if $Ri<0.7Ri_c^*$; blend if in band; Ri-closure if $Ri>1.3Ri_c^*$; hysteresis on/off per prior state.
\item Choose intervention: mixing-length reduction (McNider) or K-multiplier (Biazar). Preserve $2\Delta$ by enforcing $G(0)=1$ and $G'(0)=0$.
\end{enumerate}

\section*{Validation metrics and targets}
\begin{itemize}
\item Bias ratio $B$ reduction: target 40\%+ reduction at $\Delta z=60$--100\,m.
\item Surface flux RMSE reduction (tower/LES): target $<$10--15\,W\,m$^{-2}$ improvement.
\item Regime classification accuracy vs manual/LES labels: $>$85\%.
\item Computational overhead: $<$5\% for operational module.
\end{itemize}

\section*{Timeline (proposed)}
\begin{itemize}
\item 0--2 weeks: diagnostic runs and compute Ri\_c* statistics from archived WRF/LES outputs (Biazar assist with I/O).
\item 2--8 weeks: McNider proposes candidate Ri\_c* forms and l\_mod functions; Biazar implements K-path prototype in Python/CMAQ branch.
\item 8--20 weeks: calibration (Bayesian/optimization) on selected datasets; offline 1D column tests and LES comparisons.
\item 20--36 weeks: integrate into WRF/CMAQ, regression tests and case studies; prepare manuscript and code release.
\end{itemize}

\section*{Deliverables}
\begin{itemize}
\item Parameterized Ri\_c* functional forms (analytic + code).
\item Two intervention modules: (A) mixing-length updater; (B) K-multiplier — both with unit tests preserving $2\Delta$.
\item Validation report: tower/LES/urban case studies and computational-cost assessment.
\item Draft manuscript for BLM/MWR and GitHub release with examples and namelist options.
\end{itemize}

\section*{Immediate next steps for you (one-line)}
Please confirm which intervention (mixing-length first vs K-multiplier first) you prefer to prototype; McNider to deliver physics forms, Biazar to prepare operational hooks — target diagnostic runs in 2 weeks.

\vspace{6pt}
\noindent Respectfully,\\
Project lead (contact in repo)

\end{document}